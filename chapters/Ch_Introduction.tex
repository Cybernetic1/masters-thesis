\chapter{Introduction}\label{chap:introduction}

\section{Background}

Lambek:
\begin{equation}
\begin{tikzcd}[row sep=1cm, column sep=0.6cm]
& \parbox{1.5cm}{\linespread{-0.5}\selectfont\centering type theory} \arrow[dl,dash] \arrow[dr,dash] & \\
\mbox{logic} \arrow[rr,dash] & {} & \parbox{1.5cm}{\linespread{-0.5}\selectfont\centering category theory}
\end{tikzcd}
\end{equation}

I am very curious to see if the two algebras below would coincide?
\begin{equation}
\begin{tikzcd}[row sep=1cm, column sep=0.6cm]
 & & \parbox{1.5cm}{\linespread{-0.5}\selectfont\centering type theory} \arrow[dl,dash] \arrow[dr,dash] & & & \\
\parbox{2cm}{\linespread{-0.5}\selectfont\centering algebraic logic} \arrow[rrrrr, dashed, dash, bend right] \arrow[r,dash] & \mbox{logic} \arrow[rr,dash] & {} & \parbox{1.5cm}{\linespread{-0.5}\selectfont\centering category theory} \arrow[r,dash] & \mbox{topos} \arrow[r,dash] & \parbox{2cm}{\linespread{-0.5}\selectfont\centering algebraic geometry}
\end{tikzcd}
\end{equation}

\section{Paul Halmos}

Every Boolean algebra $\mathbb{A}$ is isomorphic to the set of all continuous functions from $X$ into $\mathbb{O}$, where $X$ is the dual space of the algebra $\mathbb{A}$, and $\mathbb{O}$ is the Boolean algebra with 2 elements.  If there is a homomorphism $f$ between Boolean algebras $\mathbb{A} \rightarrow \mathbb{B}$ then there is a dual morphism $f^*$ between their dual spaces $Y \rightarrow X$:
\begin{equation}
\begin{tikzcd}[column sep=3cm, row sep=0.6cm]
\mathbb{A} \arrow[r,"f"] \arrow[d,shift left=2,phantom,"\cong"{anchor=south,rotate=90}] & \mathbb{B} \arrow[d,shift left=2,phantom,"\cong"{anchor=south,rotate=90}] \\
\overbracket{X} \arrow[d] & \overbracket{Y} \arrow[d] \arrow[l,"f^*"] \\
\underbracket{\mathbb{O}} & \underbracket{\mathbb{O}} 
\end{tikzcd}
\end{equation}

\section{The set-up}

The set of equations $F$ defines an algebraic set = \textbf{the world}:
\begin{equation}
F(x) = 0 .
\end{equation}
The objective of an intelligent agent is to learn $F$.

We have the function $f$ performing \textbf{prediction} of the immediate future:
\begin{equation}
\boxed{\mbox{current state}} \quad x_t \stackrel{f}{\mapsto} x_{t+1} \quad \boxed{\mbox{next state}} \;.
\end{equation}

In an infinitesimal sense, we can see $f$ as a \textbf{differential equation} describing the \textbf{world trajectory}:
\begin{equation}
\dot{x} = f(x) .
\end{equation}
So $F$ is the \textbf{solution} to this differential equation.

It seems that $F$ and $f$ are more or less equivalent ways to describe the world.

Logic can be turned into some form of algebra, and this algebra can be used to express either $F$ or $f$.  Perhaps both ways are feasible, or even mixing the two.

What does it mean to use logic to express $F$ or $f$?

\begin{equation}
\begin{tabular}{|c|c|c|}
	\hline
	\textbf{LOGIC} & \textbf{facts} & \textbf{rules} \\
		& \mbox{human(socrates)} & $\forall x. \mbox{human}(x) \rightarrow \mbox{mortal}(x)$ \\
	\hline
	\textbf{ALGEBRA} & \textbf{element} & \textbf{element} \\
		& $p \in \mathbb{A} $ & $(p \rightarrow q) \in \mathbb{A} $ \\
	\hline
	\textbf{WORLD} & \textbf{states} & \textbf{state transitions} \\
		& $x_t$ & $x_t \stackrel{f}{\mapsto} x_{t+1}$ \\
	\hline
\end{tabular}
\end{equation}
