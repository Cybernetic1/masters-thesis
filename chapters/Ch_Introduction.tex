\chapter{Introduction}\label{chap:introduction}

\section{The set-up}

The set of equations $F$ defines an algebraic set = \textbf{the world}:
\begin{equation}
F(x) = 0 .
\end{equation}
The objective of an intelligent agent is to learn $F$.

We have the function $f$ performing \textbf{prediction} of the immediate future:
\begin{equation}
\mbox{current state} \stackrel{f}{\mapsto} \mbox{next state} .
\end{equation}

In an infinitesimal sense, we can see $f$ as a \textbf{differential equation} describing the \textbf{world trajectory}:
\begin{equation}
\dot{x} = f(x) .
\end{equation}
So $F$ is the \textbf{solution} to this differential equation.

It seems that $F$ and $f$ are more or less equivalent ways to describe the world.

Logic can be turned into some form of algebra, and this algebra can be used to express either $F$ or $f$.  Perhaps both ways are feasible, or even mixing the two.

What does it mean to use logic to express $F$ or $f$?

